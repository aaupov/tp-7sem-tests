\section*{Тест 3}
\begin{enumerate}
    \item Эволюция стационарного состояния
        $$\ket{n,t}=e^{-\frac{i E_n t}{\hbar}}\ket{n}$$
    \item Связь координатного и импульсного представления состояний
        
$$\braket{q|\psi}=\frac{1}{\sqrt{2\pi\hbar}}\int^{+\infty}_{-\infty}dp\; e^\frac{ipq}{\hbar}\braket{p|\psi}$$
    \item Что гарантирует полноту системы стационарных состояний

        $\bar H$ явно не зависит от времени
    \item Принцип суперпозиции

        Если система может находиться в состояниях $\ket{\psi_1}$ и $\ket{\psi_2}$,
        то она может находиться и в состоянии $C_1\ket{\psi_1}+C_2\ket{\psi_2},\quad C_1,C_2\in \mathbb C$
    \item Собственное состояние оператора координаты в импульсном представлении
        $$\braket{p|q}=\frac{1}{\sqrt{2\pi\hbar}}e^{-\frac{ipq}{\hbar}}$$
    \item Гамильтониан осциллятора через $\hat p$ и $\hat q$

        $$\hat H=\frac{{\hat P}^2}{2m}+\frac{m\omega^2\hat Q^2}{2}$$
    \item Стационарное уравнение Шредингера как математическая задача

        Задача на нахождение полной системы собственных векторов и собственных значений
    \item $[\hat q^{n+1},\hat p]={\hat q}^ni\hbar + [\hat q^n,\hat p]\hat q$
    \item Что такое спектр оператора

        Совокупность собственных чисел оператора
    \item Общий вид решения уравнения Шредингера
        $$\ket{\psi,t}=\sum_{n\in\mathbb Q} C_n e^{-\frac{iE_n t}{\hbar}}\ket{n}$$
\end{enumerate}
