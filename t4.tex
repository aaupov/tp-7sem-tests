\section{Тест 4}
\begin{enumerate}
    \item $(\hat A\otimes\hat B)(\hat C\otimes\hat D)\ket{\alpha}\ket{\beta} = \hat A\hat C\ket{\alpha}\hat B\hat D\ket{\beta}$
    \item $[\hat p_\alpha,\hat q_\beta]=i\hbar e_{\beta\alpha\gamma}\hat p_\gamma $
    \item Какие операторы м.к.д. коммутируют
        $$\hat{\bar L}^2; \hat L_z$$
    \item Def эрмитова сопряжения
        $$\exists F^\dagger \colon \:\forall \ket{y},\ket{\psi}\quad \braket{y|\hat F\psi}=\braket{\hat F^\dagger y|\psi} $$
    \item $[\hat L_\alpha,\hat L_\beta]=i\hbar e_{\alpha\beta\gamma}\hat L_\gamma$
    \item Def унитарного оператора
        $$\braket{\hat U y|\hat U\psi}\equiv\braket{y|\psi}, \; \forall \ket{y} ,\ket{\psi}\Rightarrow \hat U-\hbox{унитарный}\;$$
    \item $\braket{\bar r|\hat l_z|\psi}=-i\frac{\partial}{\partial y} \braket{\bar r|\psi}$
        \item Коэффициенты разложения $\ket{\psi}$ по базису $\ket{\alpha}$
            $$C_\alpha=\braket{\alpha|\psi}$$
    \item Def стационарного состояния

        Состояние, собственное для $\hat H$
    \item Def одновременной измеримости
        \begin{align*}
            \hat A\;& a_1\\
            \hat B\;& b_1\\
            \hat C\;& a_2\\
        \end{align*}
        Две величины называются одновременно измеримыми, если $a_1=a_2$ всегда 
        при многократном повторе
\end{enumerate}
