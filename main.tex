\documentclass{article}
\usepackage{polyglossia}   %% загружает пакет многоязыковой вёрстки
    \setdefaultlanguage[spelling=modern]{russian}  %% устанавливает главный язык документа
    \setotherlanguage{english} %% объявляет второй язык документа
\defaultfontfeatures{Ligatures={TeX},Renderer=Basic}  %% свойства шрифтов по умолчанию
%\setmainfont[Ligatures={TeX,Historic}]{CMU Serif} %% задаёт основной шрифт документа
\setmainfont[Ligatures={TeX,Historic}]{PT Serif} %% задаёт основной шрифт документа
%\setsansfont{CMU Sans Serif}                    %% задаёт шрифт без засечек
%\setmonofont{CMU Typewriter Text}               %% задаёт моноширинный шрифт
\setsansfont{PT Sans}                    %% задаёт шрифт без засечек
\setmonofont{PT Mono}               %% задаёт моноширинный шрифт
%\newfontfamily\cyrillicfont{CMU Serif}
\pagestyle{empty}

\usepackage[intlimits]{amsmath}
\usepackage{amssymb}
\usepackage{braket}
\usepackage{eqnarray}
\usepackage{xunicode}

\newcommand{\eqdef}{\overset{\mathrm{def}}{=\joinrel=}}

\begin{document}
\section{Тест 1}
\begin{enumerate}
    \item Def унитарного оператора
        $$\braket{\hat U y|\hat U\psi}\equiv\braket{y|\psi}, \; \forall \ket{y} ,\ket{\psi}\Rightarrow \hat U-\hbox{унитарный}\;$$
    \item Свойства эрмитовых операторов

        Собственные значения эрмитовых операторов действительны
        $$\hat F\ket{\alpha}=f_\alpha \ket{\alpha} \Rightarrow f_\alpha \in \mathbb R$$
    \item Два собственных вектора, соответствующие различным собственным
        значениям, ортогональны
    \item Система собственных векторов эрмитова оператора полна
    \item Что такое полнота базиса
        
        Любой вектор можно разложить по базисным векторам, причем единственным
        образом
    \item Операторное разложение $\hat 1$
        $$\hat 1=\sum_{\alpha\in\mathbb Q}\ket{\alpha}\bra{\alpha}$$
    \item Антиэрмитов оператор 
       $$\hat F^\dagger=-\hat F$$
   \item Def коммутатора $$[\hat A,\hat B]=\hat A\hat B-\hat B\hat A$$
    \item Полная система коммутирующих операторов

        Любому набору собственных значений операторов соответствует один 
        собственный вектор
    \item Def эрмитова сопряжения
        $$\braket{y|\hat F\psi}=\braket{\hat F^\dagger y|\psi}=\braket{\psi|\hat F^\dagger y}^*,\quad \forall\ket{y},\ket{\psi}$$
\end{enumerate}

\section*{Тест 2}
\begin{enumerate}
    \item ?
    \item $[\hat A,\hat B]^+=-[\hat A^+,\hat B^+]$
    \item Оператор импульса в импульсном представлении
        $$\hat p \hat T\bra{p}=p$$
    \item Гамильтониан осциллятора через $\hat a$ и $\hat a^+$
        $$\hat H=\hbar\omega(\hat a^+\hat a+\frac{1}{2})$$
    \item Полная система комм. наблюдаемых

        Любому набору собственных значений соответствует только один 
        собственный вектор
    \item Оператор координаты в импульсном представлении
        $$\hat q\hat T\bra{p}=i\hbar\frac{\partial}{\partial p}$$
    \item Временное уравнение Шредингера как математическая задача

        Задача Коши
        %$$i\hbar\frac{\partial}{\partial t}\ket{\psi,t}=\hat H\ket{\psi,t} ...?$$
    \item Операторное разложение $\hat 1$
        $$\hat 1=\sum_{\alpha\in\mathbb Q}\ket{\alpha}\bra{\alpha}$$
        %$$\hat 1=\int dp\;\ket{p}\bra{p}$$
    \item ?
    \item ?
\end{enumerate}

\section{Тест 3}
\begin{enumerate}
    \item Эволюция стационарного состояния
        $$\ket{n,t}=e^{-\frac{i E_n t}{\hbar}}\ket{n}$$
    \item Связь координатного и импульсного представления состояний
        
$$\braket{q|\psi}=\frac{1}{\sqrt{2\pi\hbar}}\int^{+\infty}_{-\infty}dp\; e^\frac{ipq}{\hbar}\braket{p|\psi}$$
    \item Что гарантирует полноту системы стационарных состояний

        $\bar H$ явно не зависит от времени
    \item Принцип суперпозиции

        Если система может находиться в состояниях $\ket{\psi_1}$ и $\ket{\psi_2}$,
        то она может находиться и в состоянии $C_1\ket{\psi_1}+C_2\ket{\psi_2},\quad C_1,C_2\in \mathbb C$
    \item Собственное состояние оператора координаты в импульсном представлении
        $$\braket{p|q}=\frac{1}{\sqrt{2\pi\hbar}}e^{-\frac{ipq}{\hbar}}$$
    \item Гамильтониан осциллятора через $\hat p$ и $\hat q$

        $$\hat H=\frac{{\hat P}^2}{2m}+\frac{m\omega^2\hat Q^2}{2}$$
    \item Стационарное уравнение Шредингера как математическая задача

        Нахождение чегоблять собственных векторов и собственных значений
    \item $[\hat q^{n+1},\hat p]={\hat q}^ni\hbar + [\hat q^n,\hat p]\hat q$
    \item Что такое спектр оператора

        Совокупность собственных чисел оператора
    \item Общий вид решения уравнения Шредингера
        $$\ket{\psi,t}=\sum_{n\in\mathbb Q} C_n e^{-\frac{iE_n t}{\hbar}}\ket{n}$$
\end{enumerate}

\section*{Тест 4}
\begin{enumerate}
    \item $(\hat A\otimes\hat B)(\hat C\otimes\hat D)\ket{\alpha}\ket{\beta} = \hat A\hat C\ket{\alpha}\hat B\hat D\ket{\beta}$
    \item $[\hat p_\alpha,\hat q_\beta]=i\hbar e_{\beta\alpha\gamma}\hat p_\gamma $
    \item Какие операторы м.к.д. коммутируют
        $$\hat{\bar L}^2; \hat L_z$$
    \item Def эрмитова сопряжения
        $$\exists F^\dagger \colon \:\forall \ket{y},\ket{\psi}\quad \braket{y|\hat F\psi}=\braket{\hat F^\dagger y|\psi} $$
    \item $[\hat L_\alpha,\hat L_\beta]=i\hbar e_{\alpha\beta\gamma}\hat L_\gamma$
    \item Def унитарного оператора
        $$\braket{\hat U y|\hat U\psi}\equiv\braket{y|\psi}, \; \forall \ket{y} ,\ket{\psi}\Rightarrow \hat U-\hbox{унитарный}\;$$
    \item $\braket{\bar r|\hat l_z|\psi}=-i\frac{\partial}{\partial y} \braket{\bar r|\psi}$
        \item Коэффициенты разложения $\ket{\psi}$ по базису $\ket{\alpha}$
            $$C_\alpha=\braket{\alpha|\psi}$$
    \item Def стационарного состояния

        Состояние, собственное для $\hat H$
    \item Def одновременной измеримости
        \begin{align*}
            \hat A\;& a_1\\
            \hat B\;& b_1\\
            \hat C\;& a_2\\
        \end{align*}
        Две величины называются одновременно измеримыми, если $a_1=a_2$ всегда 
        при многократном повторе
\end{enumerate}

\section*{Тест 5}
\begin{enumerate}
    \item $\Hat s_+ \ket{s,m} = \sqrt{s(s+1)-m(m+1)}\ket{s,m+1}$
    \item Def орбитального момента количества движения
        $$\hat L_\alpha = e_{\alpha\beta\gamma}\hat r_\beta \hat 
p_\gamma$$
    \item Пример оператора м.к.д. и собственных значений
$$[\hat s_\alpha;\hat s_\beta] = i\hbar e_{\alpha\beta\gamma}\hat 
s_\gamma, s\not \in \mathbb Z^+$$
    \item Минимальное собственное значение $\Hat{\bar l}^2$, если $\hat 
l_z \to -3$
$$\lambda_{min}=12$$
    \item Def когерентного состояния

        Состояние, собственное для оператора $\hat a$, оператора уничтожения
    \item Спектр гамильтониана $\hat H=\frac{a{\hat p}^2+b{\hat q}^2}{2}$
        $$E_n = \hbar\sqrt{ab}\left(n+\frac{1}{2}\right)$$
    \item Размерность спинового пространства с $\hat{\bar s}^2\to \frac{15}{4}$
        $$s(s+1)=\frac{15}{4};\quad s=\underbrace{\frac{3}{2}},-\frac{5}{2}; \qquad 2s+1=2\frac{3}{2}+1=\underline{4}$$
    \item Def оператора проектора
       %$$\Hat P\Hat P\eqdef\hat P$$
        $$\ket{\alpha}\bra{\alpha}\equiv \hat P_\alpha$$
   \item Когда $\ket{\psi}\bra{\psi}$---не проектор?
       $$\braket{\psi|\psi}\neq 1$$
   \item Какой оператор и какое собственное значение у $\ket{s,-s}$
       \begin{eqnarray}
           {\Hat{\bar s}}^2\ket{s,-s}&=&s(s+1)\ket{s,-s}\\
           \hat s_z\ket{s,-s}&=&-s\ket{s,-s}
       \end{eqnarray}
\end{enumerate}

\section{Тест 6}
\begin{enumerate}
    \item Собственные значения проекции М.К.Д для $s=\frac{9}{2}$
        $$-\frac{9}{2};-\frac{7}{2};-\frac{5}{2};-\frac{3}{2};-\frac{1}{2}
        -\frac{1}{2};-\frac{3}{2};-\frac{5}{2};-\frac{7}{2};-\frac{9}{2}$$
    \item Def оператора инверсии
        $$\hat I\colon 
	\Braket{\bar r | \hat I | \psi} = \Braket{-\bar r | \psi}$$
    \item Собственное значение квадрата М.К.Д. $\frac{9}{2}$
        $$\frac{99}{4}$$
    \item Def интеграла движения
        $$\frac{d\hat F}{dt}=0$$
    \item Def центрально-симметричного поля
        $$u(\bar r)\eqdef u(r)$$
    \item Для каких операторов сферическая функция является собственной
        $$\Hat L_z; \Hat{\bar L}^2$$
    \item Полная производная по времени оператора
        $$\Braket{\psi|\frac{d\hat F}{dt}|\varphi}\equiv\frac{d}{dt}\braket{\psi|\hat F|\varphi}$$
    \item Полный набор коммутирующих наблюдаемых центра симметрии
        $$\hat H; \hat L_z; \Hat{\bar L} ^2$$
    \item Уравнение на интеграл движения
        $$\frac{\partial\hat F}{\partial t}+\frac{i}{\hbar}[\hat H;\hat F] = 0$$
    \item $[\hat L_\alpha; \hat P_\beta \hat P_\beta]=0$
\end{enumerate}

\end{document}
