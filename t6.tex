\section{Тест 6}
\begin{enumerate}
    \item Собственные значения проекции М.К.Д для $s=\frac{9}{2}$
        $$-\frac{9}{2};-\frac{7}{2};-\frac{5}{2};-\frac{3}{2};-\frac{1}{2}
        -\frac{1}{2};-\frac{3}{2};-\frac{5}{2};-\frac{7}{2};-\frac{9}{2}$$
    \item Def оператора инверсии
        $$\hat I\colon 
	\Braket{\bar r | \hat I | \psi} = \Braket{-\bar r | \psi}$$
    \item Собственное значение квадрата М.К.Д. $\frac{9}{2}$
        $$\frac{99}{4}$$
    \item Def интеграла движения
        $$\frac{d\hat F}{dt}=0$$
    \item Def центрально-симметричного поля
        $$u(\bar r)\eqdef u(r)$$
    \item Для каких операторов сферическая функция является собственной
        $$\Hat L_z; \Hat{\bar L}^2$$
    \item Полная производная по времени оператора
        $$\Braket{\psi|\frac{d\hat F}{dt}|\varphi}\equiv\frac{d}{dt}\braket{\psi|\hat F|\varphi}$$
    \item Полный набор коммутирующих наблюдаемых центра симметрии
        $$\hat H; \hat L_z; \Hat{\bar L} ^2$$
    \item Уравнение на интеграл движения
        $$\frac{\partial\hat F}{\partial t}+\frac{i}{\hbar}[\hat H;\hat F] = 0$$
    \item $[\hat L_\alpha; \hat P_\beta \hat P_\beta]=0$
\end{enumerate}
