\section{Тест 1}
\begin{enumerate}
    \item Def унитарного оператора
        $$\braket{\hat U y|\hat U\psi}\equiv\braket{y|\psi}, \; \forall \ket{y} ,\ket{\psi}\Rightarrow \hat U-\hbox{унитарный}\;$$
    \item Свойства эрмитовых операторов

        Собственные значения эрмитовых операторов действительны
        $$\hat F\ket{\alpha}=f_\alpha \ket{\alpha} \Rightarrow f_\alpha \in \mathbb R$$
    \item Два собственных вектора, соответствующие различным собственным
        значениям, ортогональны
    \item Система собственных векторов эрмитова оператора полна
    \item Что такое полнота базиса
        
        Любой вектор можно разложить по базисным векторам, причем единственным
        образом
    \item Операторное разложение $\hat 1$
        $$\hat 1=\sum_{\alpha\in\mathbb Q}\ket{\alpha}\bra{\alpha}$$
    \item Антиэрмитов оператор 
       $$\hat F^\dagger=-\hat F$$
   \item Def коммутатора $$[\hat A,\hat B]=\hat A\hat B-\hat B\hat A$$
    \item Полная система коммутирующих операторов

        Любому набору собственных значений операторов соответствует один 
        собственный вектор
    \item Def эрмитова сопряжения
        $$\braket{y|\hat F\psi}=\braket{\hat F^\dagger y|\psi}=\braket{\psi|\hat F^\dagger y}^*,\quad \forall\ket{y},\ket{\psi}$$
\end{enumerate}
